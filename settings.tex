
\usepackage{geometry}
\usepackage{graphicx}
\usepackage{amssymb}
\usepackage{color}          	% gives color options
\usepackage{url}		% produces hyperlinks
\usepackage[english]{babel}
\usepackage{colortbl}	% allows for color usage in tables
\usepackage{multirow}	% allows for rows that span multiple rows in tables
\usepackage{xcolor}		% this package has a variety of color options
\usepackage{calc}
\usepackage{multicol}
\usepackage{wrapfig}
\usepackage{textcomp}
\usepackage{bm}
\usepackage{bbm}
\usepackage{setspace}
\usepackage{changepage}
\usepackage{isotope}
\singlespacing

\usepackage{fontspec}
\newfontfamily\DejaSans{DejaVu Sans}

%%%%%%%%%%%%%%%%
% Small code output
%%%%%%%%%%%%%%%%

%% change fontsize of R code

\makeatletter
\@ifundefined{Shaded}{\newenvironment{Shaded}{}{}}{}
\makeatother


\let\oldShaded\Shaded
\let\endoldShaded\endShaded
\renewenvironment{Shaded}{\footnotesize\begin{spacing}{0.9}\oldShaded}{\endoldShaded\end{spacing}}

%% change fontsize of output
\let\oldverbatim\verbatim
\let\endoldverbatim\endverbatim
\renewenvironment{verbatim}{\footnotesize\begin{spacing}{0.9}\oldverbatim}{\endoldverbatim\end{spacing}}


\newcommand{\tinyoutput}{
  \renewenvironment{Shaded}{\tiny\begin{spacing}{0.9}\oldShaded}{\endoldShaded\end{spacing}}
  \renewenvironment{verbatim}{\tiny\begin{spacing}{0.9}\oldverbatim}{\endoldverbatim\end{spacing}}
}

\newcommand{\scriptoutput}{
  \renewenvironment{Shaded}{\scriptsize\begin{spacing}{0.9}\oldShaded}{\endoldShaded\end{spacing}}
  \renewenvironment{verbatim}{\scriptsize\begin{spacing}{0.9}\oldverbatim}{\endoldverbatim\end{spacing}}
}

\newcommand{\footnoteoutput}{
  \renewenvironment{Shaded}{\footnotesize\begin{spacing}{0.9}\oldShaded}{\endoldShaded\end{spacing}}
  \renewenvironment{verbatim}{\footnotesize\begin{spacing}{0.9}\oldverbatim}{\endoldverbatim\end{spacing}}
}

%\newcommand{\verbatimfont}[1]{\renewcommand{\verbatim@font}{\ttfamily#1}}


%%%%%%%%%%%%%%%%
% Custom Colors
%%%%%%%%%%%%%%%%

\definecolor{redhl}{rgb}{0.98,0.29,0.28}
\definecolor{yellowhl}{rgb}{0.98,0.87,0.28}
\newcommand{\hlr}[1]{\fcolorbox{redhl}{white}{$\displaystyle #1$}}
\newcommand{\hly}[1]{\fcolorbox{yellowhl}{white}{$\displaystyle #1$}}

\newcommand{\vvfill}{\vskip0pt plus 1filll}


\xdefinecolor{oiBlue}{rgb}{0.15, 0.35, 0.55}
\xdefinecolor{gray}{rgb}{0.5, 0.5, 0.5}
\xdefinecolor{darkGray}{rgb}{0.3, 0.3, 0.3}
\xdefinecolor{darkerGray}{rgb}{0.2, 0.2, 0.2}
\xdefinecolor{rubineRed}{rgb}{0.89,0,0.30}
\xdefinecolor{linkCol}{rgb}{0.11,0.49,0.95}	
\xdefinecolor{irishGreen}{rgb}{0,0.60,0}	
\xdefinecolor{darkturquoise}{rgb}{0.44, 0.58, 0.86}
\definecolor{lightGreen}{rgb}{0.533,0.765,0.42}
%\xdefinecolor{hlblue}{rgb}{0.051,0.65,1}
\xdefinecolor{hlblue}{rgb}{ 0.055, 0.639, 0.831}
\definecolor{light}{rgb}{.337,.608,.741}
\definecolor{dark}{rgb}{.337,.608,.741}

\definecolor{cpink}{rgb}{0.93, 0.23, 0.51}

%%%%%%%%%%%%%%%%
% Custom Commands
%%%%%%%%%%%%%%%%

% text colors
\newcommand{\red}[1]{\textit{\textcolor{rubineRed}{#1}}}
\newcommand{\orange}[1]{\textit{\textcolor{orange}{#1}}}
\newcommand{\pink}[1]{\textit{\textcolor{rubineRed!90!white!50}{#1}}}
\newcommand{\green}[1]{\textit{\textcolor{irishGreen}{#1}}}
\newcommand{\blue}[1]{\textit{\textcolor{darkturquoise}{#1}}}
\newcommand{\light}[1]{\textcolor{light}{\textbf{#1}}}
\newcommand{\dark}[1]{\textcolor{dark}{#1}}
\newcommand{\gray}[1]{\textcolor{gray}{#1}}


% links: webURL, webLin, appLink
\newcommand{\webURL}[1]{\urlstyle{same}{\textit{\textcolor{linkCol}{\url{#1}}} }}
\newcommand{\webLink}[2]{\href{#1}{\textcolor{linkCol}{{#2}}}}
\newcommand{\appLink}[2]{\href{#1}{\textcolor{lightGreen!80!black!90}{{#2}}}}

% mail
\newcommand{\mail}[1]{\href{mailto:#1}{\textit{\textcolor{linkCol}{#1}}}}

% highlighting: hl, hlGr, mathhl
\newcommand{\hl}[1]{\textit{\textcolor{hlblue}{#1}}}
\newcommand{\hlGr}[1]{\textit{\textcolor{lightGreen}{#1}}}
\newcommand{\hlRd}[1]{\textit{\textcolor{rubineRed}{#1}}}
\newcommand{\mathhl}[1]{\textcolor{hlblue}{\ensuremath{#1}}}

% example
\newcommand{\ex}[1]{\textcolor{blue}{{{\small (#1)}}}}


\DeclareMathOperator*{\argmin}{arg\,min}
\DeclareMathOperator*{\argmax}{arg\,max}